\subsection{Motivación}
La tensión de Hubble sugiere una discrepancia entre inferencias tempranas (CMB) y mediciones locales. HQCB explora un mecanismo donde el sector Higgs presenta una evolución efectiva lenta $v_{\mathrm{eff}}(t)$, inducida por acoplos efectivos a geometría cuántica discreta. El objetivo es unificar: (i) masa y ruptura electrodébil, (ii) estructura de la geometría cuántica tipo LQG/LQC, y (iii) parámetros cosmológicos observables.

\subsection{Contribuciones principales}
(i) Un formalismo mínimo basado en una acción efectiva con términos de acoplo Higgs--geometría; (ii) un puente conceptual a LQG/LQC mediante operadores discretos sobre grafos y un régimen semiclasical; (iii) un cierre bootstrap que liga escalas (por ejemplo $\Lambda$ y $v_{\mathrm{eff}}$); y (iv) un pipeline reproducible de inferencia con likelihood con covarianza (BAO) y contraste con $H_0$.

\subsection{Estrategia de validación}
El enfoque correcto para evaluar HQCB es: derivación $\rightarrow$ predicción $\rightarrow$ ajuste con incertidumbres $\rightarrow$ comparación con modelos competidores (por ejemplo $\Lambda$CDM, $w$CDM, EDE). Este repositorio implementa el pipeline reproducible y deja explícitos los puntos donde se insertan derivaciones más rigurosas.