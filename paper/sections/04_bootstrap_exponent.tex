\subsection{Cierre bootstrap}
El “bootstrap” se define como un conjunto de condiciones de auto-consistencia que ligan el sector Higgs, la geometría cuántica y parámetros cosmológicos. Un ejemplo de ansatz es:
\begin{equation}
\rho_\Lambda \propto \left(\frac{v_{\mathrm{eff}}^2}{M_{\mathrm{Pl}}^2}\right)^{\gamma},
\end{equation}
donde $\gamma$ podría tomar un valor privilegiado bajo supuestos específicos.

\subsection{El candidato $\gamma=11/3$: estatus correcto}
En formato paper, $\gamma=11/3$ debe tratarse como:
\begin{itemize}
\item \textbf{(A) Resultado derivado}: si se demuestra desde operadores y conteo de dimensiones/anomalías en un esquema definido; o
\item \textbf{(B) Parámetro inferible}: si aún no existe derivación completa, se infiere de datos (con prior informado) y se reporta sensibilidad a supuestos.
\end{itemize}
Este repositorio implementa el pipeline para (B) y deja el espacio formal para (A).

\subsection{Dependencia de esquema / universalidad}
Es crítico distinguir: (i) exponente universal (independiente de regularización/renormalización), vs. (ii) exponente efectivo que cambia con el esquema o con el truncamiento. El análisis debe incluir tests de robustez variando parametrizaciones del cierre.