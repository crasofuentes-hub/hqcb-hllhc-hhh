\subsection{Grados de libertad discretos: grafos y operadores}
En LQG, la geometría se codifica en redes de espín sobre grafos; operadores de área/volumen tienen espectro discreto. El puente HQCB requiere identificar un funcional efectivo donde el sector Higgs acople a invariantes geométricos discretizados (por ejemplo, expectativas de operadores de volumen/curvatura).

\subsection{Régimen semiclasical}
El requisito crítico es controlar $\langle \hat{O}_{\mathrm{geom}}\rangle$ en estados semiclasicales (coherentes) para recuperar FRW con correcciones controladas. La hipótesis operativa es que este valor esperado induce términos efectivos en la ecuación del Higgs (y/o en la ecuación de Friedmann) que producen un drift lento de $v_{\mathrm{eff}}$.

\subsection{LQC como laboratorio}
En LQC, las correcciones efectivas a la dinámica FRW suelen expresarse como modificaciones en $H^2$. HQCB puede usar LQC como entorno de prueba: si se postula un acoplo Higgs--curvatura/volumen efectivo, se deriva una familia de correcciones parametrizadas y se confrontan con BAO/CMB.

\subsection{Criterios de consistencia}
(i) Recuperación de GR+$\Lambda$CDM en el límite apropiado; (ii) estabilidad y ausencia de patologías (ghosts, inestabilidades); (iii) dependencia de esquema identificada; (iv) predicciones falsables claras.