\subsection{Acción efectiva mínima}
Como punto de partida, consideramos una acción efectiva del tipo:
\begin{equation}
S = \int d^4x \sqrt{-g}\left[\frac{M_{\mathrm{Pl}}^2}{2}R - \Lambda
+ \mathcal{L}_{\mathrm{SM}}(H,\psi,A_\mu) + \Delta \mathcal{L}_{\mathrm{HQCB}}\right],
\end{equation}
donde $\Delta \mathcal{L}_{\mathrm{HQCB}}$ contiene términos efectivos que capturan el acoplo entre el sector Higgs y la geometría cuántica discreta.

\subsection{Términos HQCB: ejemplo de parametrización}
Una parametrización común incluye:
\begin{equation}
\Delta \mathcal{L}_{\mathrm{HQCB}} = -\xi \, H^\dagger H\, R + \mathcal{O}\left(\frac{(H^\dagger H)^2}{M_{\mathrm{Pl}}^2}\right)
+ \mathcal{L}_{\mathrm{disc}}[g; \Gamma],
\end{equation}
donde $\Gamma$ denota grados de libertad discretos (grafo) asociados a LQG/LQC. En el repositorio, el núcleo operativo se expresa mediante una evolución efectiva $v_{\mathrm{eff}}(t)$.

\subsection{Ecuaciones de campo y régimen efectivo}
La variación de $S$ produce ecuaciones de Einstein modificadas y la ecuación efectiva del Higgs en un fondo cosmológico. El objetivo de HQCB es mostrar cómo la retroacción efectiva de $\Gamma$ induce una evolución lenta:
\begin{equation}
v_{\mathrm{eff}}(t)\approx v_0\left(\frac{t_0}{t}\right)^{\kappa},
\end{equation}
y cómo ésta se traduce en predicciones observacionales para $H(z)$ y escalas acústicas.

\subsection{Qué falta para rigor completo}
Para una derivación completa se requiere: (i) especificación de $\mathcal{L}_{\mathrm{disc}}$ desde operadores en LQG/LQC; (ii) control del límite semiclasical; (iii) renormalización/depencencia de esquema; (iv) consistencia con constraints de física de partículas.