\documentclass[11pt]{article}

\usepackage[a4paper,margin=1in]{geometry}
\usepackage{amsmath,amssymb,amsfonts}
\usepackage{physics}
\usepackage{graphicx}
\usepackage{hyperref}
\usepackage{booktabs}
\usepackage{siunitx}
\usepackage{microtype}

\title{HQCB: Higgs--Quantum Cosmological Bootstrap\\
\large Acción efectiva, puente a discretización LQG/LQC y pipeline de inferencia}
\author{Oscar Fuentes Fernández\\ \small Investigador independiente}
\date{\today}

\begin{document}
\maketitle

\begin{abstract}
Presentamos el marco HQCB (Higgs--Quantum Cosmological Bootstrap), donde el valor esperado efectivo del Higgs $v_{\mathrm{eff}}$ evoluciona lentamente debido a un acoplo efectivo con geometría cuántica discreta (LQG/LQC), cerrando un conjunto de condiciones bootstrap que ligan parámetros de partículas y cosmología. Proponemos un pipeline reproducible de inferencia con likelihood gaussiana con covarianza (BAO) y contraste con $H_0$ (tensión de Hubble), junto con predicciones falsables.
\end{abstract}

\tableofcontents

\section{Introducción}
\subsection{Motivación}
La tensión de Hubble sugiere una discrepancia entre inferencias tempranas (CMB) y mediciones locales. HQCB explora un mecanismo donde el sector Higgs presenta una evolución efectiva lenta $v_{\mathrm{eff}}(t)$, inducida por acoplos efectivos a geometría cuántica discreta. El objetivo es unificar: (i) masa y ruptura electrodébil, (ii) estructura de la geometría cuántica tipo LQG/LQC, y (iii) parámetros cosmológicos observables.

\subsection{Contribuciones principales}
(i) Un formalismo mínimo basado en una acción efectiva con términos de acoplo Higgs--geometría; (ii) un puente conceptual a LQG/LQC mediante operadores discretos sobre grafos y un régimen semiclasical; (iii) un cierre bootstrap que liga escalas (por ejemplo $\Lambda$ y $v_{\mathrm{eff}}$); y (iv) un pipeline reproducible de inferencia con likelihood con covarianza (BAO) y contraste con $H_0$.

\subsection{Estrategia de validación}
El enfoque correcto para evaluar HQCB es: derivación $\rightarrow$ predicción $\rightarrow$ ajuste con incertidumbres $\rightarrow$ comparación con modelos competidores (por ejemplo $\Lambda$CDM, $w$CDM, EDE). Este repositorio implementa el pipeline reproducible y deja explícitos los puntos donde se insertan derivaciones más rigurosas.

\section{Formalismo mínimo: acción y ecuaciones de campo}
\subsection{Acción efectiva mínima}
Como punto de partida, consideramos una acción efectiva del tipo:
\begin{equation}
S = \int d^4x \sqrt{-g}\left[\frac{M_{\mathrm{Pl}}^2}{2}R - \Lambda
+ \mathcal{L}_{\mathrm{SM}}(H,\psi,A_\mu) + \Delta \mathcal{L}_{\mathrm{HQCB}}\right],
\end{equation}
donde $\Delta \mathcal{L}_{\mathrm{HQCB}}$ contiene términos efectivos que capturan el acoplo entre el sector Higgs y la geometría cuántica discreta.

\subsection{Términos HQCB: ejemplo de parametrización}
Una parametrización común incluye:
\begin{equation}
\Delta \mathcal{L}_{\mathrm{HQCB}} = -\xi \, H^\dagger H\, R + \mathcal{O}\left(\frac{(H^\dagger H)^2}{M_{\mathrm{Pl}}^2}\right)
+ \mathcal{L}_{\mathrm{disc}}[g; \Gamma],
\end{equation}
donde $\Gamma$ denota grados de libertad discretos (grafo) asociados a LQG/LQC. En el repositorio, el núcleo operativo se expresa mediante una evolución efectiva $v_{\mathrm{eff}}(t)$.

\subsection{Ecuaciones de campo y régimen efectivo}
La variación de $S$ produce ecuaciones de Einstein modificadas y la ecuación efectiva del Higgs en un fondo cosmológico. El objetivo de HQCB es mostrar cómo la retroacción efectiva de $\Gamma$ induce una evolución lenta:
\begin{equation}
v_{\mathrm{eff}}(t)\approx v_0\left(\frac{t_0}{t}\right)^{\kappa},
\end{equation}
y cómo ésta se traduce en predicciones observacionales para $H(z)$ y escalas acústicas.

\subsection{Qué falta para rigor completo}
Para una derivación completa se requiere: (i) especificación de $\mathcal{L}_{\mathrm{disc}}$ desde operadores en LQG/LQC; (ii) control del límite semiclasical; (iii) renormalización/depencencia de esquema; (iv) consistencia con constraints de física de partículas.

\section{Puente a LQG/LQC: discretización, operadores y semiclasicalidad}
\subsection{Grados de libertad discretos: grafos y operadores}
En LQG, la geometría se codifica en redes de espín sobre grafos; operadores de área/volumen tienen espectro discreto. El puente HQCB requiere identificar un funcional efectivo donde el sector Higgs acople a invariantes geométricos discretizados (por ejemplo, expectativas de operadores de volumen/curvatura).

\subsection{Régimen semiclasical}
El requisito crítico es controlar $\langle \hat{O}_{\mathrm{geom}}\rangle$ en estados semiclasicales (coherentes) para recuperar FRW con correcciones controladas. La hipótesis operativa es que este valor esperado induce términos efectivos en la ecuación del Higgs (y/o en la ecuación de Friedmann) que producen un drift lento de $v_{\mathrm{eff}}$.

\subsection{LQC como laboratorio}
En LQC, las correcciones efectivas a la dinámica FRW suelen expresarse como modificaciones en $H^2$. HQCB puede usar LQC como entorno de prueba: si se postula un acoplo Higgs--curvatura/volumen efectivo, se deriva una familia de correcciones parametrizadas y se confrontan con BAO/CMB.

\subsection{Criterios de consistencia}
(i) Recuperación de GR+$\Lambda$CDM en el límite apropiado; (ii) estabilidad y ausencia de patologías (ghosts, inestabilidades); (iii) dependencia de esquema identificada; (iv) predicciones falsables claras.

\section{Cierre bootstrap y el exponente $\gamma$ (candidato $11/3$)}
\subsection{Cierre bootstrap}
El “bootstrap” se define como un conjunto de condiciones de auto-consistencia que ligan el sector Higgs, la geometría cuántica y parámetros cosmológicos. Un ejemplo de ansatz es:
\begin{equation}
\rho_\Lambda \propto \left(\frac{v_{\mathrm{eff}}^2}{M_{\mathrm{Pl}}^2}\right)^{\gamma},
\end{equation}
donde $\gamma$ podría tomar un valor privilegiado bajo supuestos específicos.

\subsection{El candidato $\gamma=11/3$: estatus correcto}
En formato paper, $\gamma=11/3$ debe tratarse como:
\begin{itemize}
\item \textbf{(A) Resultado derivado}: si se demuestra desde operadores y conteo de dimensiones/anomalías en un esquema definido; o
\item \textbf{(B) Parámetro inferible}: si aún no existe derivación completa, se infiere de datos (con prior informado) y se reporta sensibilidad a supuestos.
\end{itemize}
Este repositorio implementa el pipeline para (B) y deja el espacio formal para (A).

\subsection{Dependencia de esquema / universalidad}
Es crítico distinguir: (i) exponente universal (independiente de regularización/renormalización), vs. (ii) exponente efectivo que cambia con el esquema o con el truncamiento. El análisis debe incluir tests de robustez variando parametrizaciones del cierre.

\section{Fenomenología cosmológica: $H(z)$, BAO y CMB (predicciones falsables)}
\subsection{Efectos en $H(z)$ y escalas acústicas}
Una evolución lenta de $v_{\mathrm{eff}}$ puede reflejarse en cambios en la escala acústica $r_d$ (vía física temprana) y en distancias BAO (vía dinámica tardía efectiva). El repositorio incluye una implementación toy y un likelihood BAO con covarianza (mock), diseñado para ser reemplazado por datos reales.

\subsection{Predicciones falsables}
Las predicciones deben expresarse como funciones de parámetros HQCB: $H(z)$, $D_A(z)$, $D_V(z)$, shifts en BAO, impacto en CMB (p. ej. $r_s$, $z_*$), y consistencia con constraints de HL-LHC (si HQCB impacta el sector Higgs medible).

\section{Inferencia estadística: likelihood con covarianza y comparación con modelos}
\subsection{Likelihood gaussiana con covarianza}
Para BAO se usa una likelihood multivariada:
\begin{equation}
\log \mathcal{L} = -\frac{1}{2}\Delta^T C^{-1}\Delta - \frac{1}{2}\log\left|2\pi C\right|,
\end{equation}
donde $\Delta = y_{\mathrm{obs}}-y_{\mathrm{pred}}$ y $C$ es la covarianza. El repositorio implementa esta estructura y la integra con un posterior toy para $\gamma$ y $H_0$.

\subsection{Comparación con modelos competidores}
El estándar académico exige comparar HQCB contra $\Lambda$CDM, $w$CDM y/o EDE usando métricas (AIC/BIC), Bayes factors o evidencias aproximadas. Este repositorio deja preparado el pipeline; el siguiente paso es conectar datasets reales y modelos base.

\section{Discusión, limitaciones y roadmap}
\subsection{Limitaciones actuales (explícitas)}
(i) El puente LQG/LQC está codificado como parametrización efectiva; (ii) el cierre bootstrap (incluyendo $\gamma$) aún requiere derivación completa para reclamar universalidad; (iii) el dataset BAO actual es mock con covarianza, diseñado para sustituirse por datos reales.

\subsection{Roadmap inmediato}
(1) Sustituir BAO mock por datos reales (BOSS/eBOSS/DESI) con covarianza; (2) añadir comparación $\Lambda$CDM/$w$CDM/EDE; (3) derivar o restringir $\gamma$ desde un esquema LQG/LQC explícito; (4) ampliar a constraints del sector Higgs (HL-LHC) si aplica.

\bibliographystyle{unsrt}
\bibliography{bib/references}

\end{document}